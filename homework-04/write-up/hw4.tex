\documentclass[12pt]{article}
\usepackage{amsmath}
\usepackage{graphicx}

%\setlength\parindent{20pt} %% Do not touch this

\title{Homework 2} 

\author{Geneva Porter\\ 
MATH-693B Numerical Partial Differential Equations\\}

\date{February 27, 2020} 
\begin{document}
\maketitle

\section*{1.3.2}

Solve the system
$$
\begin{aligned}
u_{t}+\frac{1}{3}(t-2) u_{x}+\frac{2}{3}(t+1) w_{x}+\frac{1}{3} u &=0 \\
w_{t}+\frac{1}{3}(t+1) u_{x}+\frac{1}{3}(2 t-1) w_{x}-\frac{1}{3} w &=0
\end{aligned}
$$
by the Lax-Fricdrichs scheme: i.e., each time derivative is approximated as it is for the scalar equation and the spatial derivatives are approximated by central differences. The initial values are
$$
\begin{aligned}
u(0, x) &=\max (0,1-|x|) \\
w(0, x) &=\max (0,1-2|x|)
\end{aligned}
$$
Consider values of $x$ in [-3,3] and $t$ in $[0,2] .$ Take $h$ equal to $1 / 20$ and $\lambda$ equal
to $1 / 2 .$ At each boundary set $u=0,$ and set $w$ equal to the newly computed value one grid point in from the boundary. Describe the solution behavior for $t$ in the range $[1.5,2] .$ You may find it convenient to plot the solution. Solve the system in the form given; do not attempt to diagonalize it.


\subsubsection*{Solution}

The Lax-Friedrichs Scheme is given by:

$$ u_t = \frac{u^{n+1}_m - \frac{1}{2}\left(u^n_{m+1}+u^n_{m-1}\right)}{k}, ~~~~~~~ u_x = \frac{u_{m+1}^n - u_{m-1}^n}{2h}$$

We will use $k=1/40$ and $h=1/20$



\section*{3.2.4}

Using the box scheme $(3.2 .3),$ solve the one-way wave equation
$$
u_{t}+u_{x}=\sin (x-t)
$$
on the interval [0,1] for $0 \leq t \leq 1.2$ with $u(0, x)=\sin x$ and with $u(t, 0)=$ $-(1+t) \sin t$ as the boundary condition.

Demonstrate the second-order accuracy of the solution using $\lambda=1.2$ and $h=$ $\frac{1}{10}, \frac{1}{20}, \frac{1}{40},$ and $\frac{1}{80} .$ Measure the error in the $L^{2}$ norm $(3.1 .24)$ and the maximum norm. To implement the box scheme note that $v_{0}^{n+1}$ is given by the boundary data, and then each value of $v_{m+1}^{n+1}$ can be determined from $v_{m}^{n+1}$ and the other values.

\end{document}