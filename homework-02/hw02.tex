\documentclass[12pt]{article}
\usepackage{amsmath}
\usepackage{graphicx}

%\setlength\parindent{20pt} %% Do not touch this

\title{Homework 2} 

\author{Geneva Porter\\ 
MATH-693B Numerical Partial Differential Equations\\}

\date{February 27, 2020} 
\begin{document}
\maketitle

\section*{2.1.4}
Use an argument similar to that used in $(2.1 .11)$ to show that the initial value problem for the equation $u_{t}=u_{x x x}$ is well-posed.

\subsubsection*{Solution}

Definition 1.5.2 states that \textit{The initial value problem for the first-order partial differential equation $Pu=0$ is well-posed if for any time T $\geq 0$, there is a constant $C_T$ such that any solution $u(t,x)$ satisfies}

$$ \int_{-\infty}^\infty |u(t,x)|^2~dx \leq C_T\int_{-\infty}^\infty |u(0,x)|^2~dx $$

\noindent We can extend this to the third order partial differential equation given above. We first transform only the spatial variable, yielding:

$$ \hat{u}_t = (i\omega)^3\hat{u} = -i\omega^3\hat{u}$$

\noindent Using the initial data, the solution must be:

$$ \hat{u}(x,\omega) = e^{-i\omega^3t}\hat{u}_0(\omega) $$

\noindent Like in Strikwerda's example, we use $|e^{-i\omega^3t}|=1$ and Parseval's relation to show:

\begin{equation*}
    \begin{aligned}
    \int_{-\infty}^\infty |u(t,x)|^2~dx & = \int_{-\infty}^\infty |\hat{u}(t,\omega)|^2~d\omega \\
    ~ &= \int_{-\infty}^\infty |e^{-i\omega^3t}\hat{u}(0, \omega)|^2~d\omega \\
    ~ &= \int_{-\infty}^\infty |\hat{u}(0, \omega)|^2~d\omega \\
    ~ &= \int_{-\infty}^\infty |u(0, x)|^2~dx ~~~~ \longrightarrow \\
    \int_{-\infty}^\infty |u(t,x)|^2~dx & \leq C_T\int_{-\infty}^\infty |u(0,x)|^2~dx ~~~ \forall ~C_T \geq 1
    \end{aligned}
\end{equation*}

\noindent Thus, the initial value problem is considered well-posed.

\section*{2.1.5}
Use an argument similar to that used in $(2.1 .11)$ to show that the initial value problem for the equation $u_{t}+u_{x}+b u=0$ is well-posed.

\subsubsection*{Solution}

This solution follows from 2.1.4 in a similar fashion. First, we transform the spatial components:

$$ \hat{u}_t = -u_x-bu = -i\omega\hat{u} -b \hat{u} = -(i\omega + b)\hat{u}$$

\noindent Using the initial data, the solution must be:

$$ \hat{u}(x,\omega) = e^{-(i\omega + b)t}\hat{u}_0(\omega) $$

\noindent Now using $|e^{-(i\omega + b)t}|=1$ and Parseval's relation gives us:

\begin{equation*}
    \begin{aligned}
    \int_{-\infty}^\infty |u(t,x)|^2~dx & = \int_{-\infty}^\infty |\hat{u}(t,\omega)|^2~d\omega \\
    ~ &= \int_{-\infty}^\infty |e^{-(i\omega + b)t}\hat{u}(0, \omega)|^2~d\omega \\
    ~ &= \int_{-\infty}^\infty |\hat{u}(0, \omega)|^2~d\omega \\
    ~ &= \int_{-\infty}^\infty |u(0, x)|^2~dx ~~~~ \longrightarrow \\
    \int_{-\infty}^\infty |u(t,x)|^2~dx & \leq C_T\int_{-\infty}^\infty |u(0,x)|^2~dx ~~~ \forall ~C_T \geq 1
    \end{aligned}
\end{equation*}

\noindent Again, the initial value problem is considered well-posed.

\section*{2.2.1}
Show that the backward-time central-space scheme (1.6.1) is consistent with equation $(1.1 .1)$ and is unconditionally stable.

\subsubsection*{Solution}

The backward-time central-space scheme for $u_t+au_x=0$ is given by:

$$ \frac{u_{m}^{n+1}-u_{m}^{n}}{k} + a\frac{u_{m+1}^{n+1}-u_{m-1}^{n+1}}{2h} = 0 $$

\noindent Separating the $n$ and $n+1$ terms, we get:

$$ u^{n+1}_{m} + \frac{a\lambda}{2}u^{n+1}_{m+1} - \frac{a\lambda}{2}u^{n+1}_{m-1} = u^{n}_{m} $$

To show consistency and stability, we first use the Fourier transform on each of the terms $u^{n+1}_{m},u^{n}_{m},u^{n+1}_{m+1},u^{n+1}_{m-1}$.

\begin{equation*}
    \begin{aligned}
    \hat{u}^{n+1}_{m} &= \frac{1}{\sqrt{2\pi}}\int_{-\pi/h}^{\pi/h}e^{imh\xi}\hat{u}^{n+1}(\xi)~d\xi\\
    \hat{u}^{n+1}_{m+1} &= \frac{1}{\sqrt{2\pi}}\int_{-\pi/h}^{\pi/h}e^{imh\xi}e^{ih\xi}\hat{u}^{n+1}(\xi)~d\xi\\
    \hat{u}^{n+1}_{m-1} &= \frac{1}{\sqrt{2\pi}}\int_{-\pi/h}^{\pi/h}e^{imh\xi}e^{-ih\xi}\hat{u}^{n+1}(\xi)~d\xi\\
    \hat{u}^{n}_{m} &= \frac{1}{\sqrt{2\pi}}\int_{-\pi/h}^{\pi/h}e^{imh\xi}\hat{u}^{n}(\xi)~d\xi
    \end{aligned}
\end{equation*}

\noindent Plugging these values into the backward-time central-space scheme, we have:
$$ \frac{1}{\sqrt{2\pi}}\int_{-\pi/h}^{\pi/h}e^{imh\xi}\hat{u}^{n+1}(\xi)~d\xi~+~ \frac{a\lambda}{2\sqrt{2\pi}}\int_{-\pi/h}^{\pi/h}e^{imh\xi}e^{ih\xi}\hat{u}^{n+1}(\xi)~d\xi ~-~$$
$$\frac{a\lambda}{2\sqrt{2\pi}}\int_{-\pi/h}^{\pi/h}e^{imh\xi}e^{-ih\xi}\hat{u}^{n+1}(\xi)~d\xi ~=~ \frac{1}{\sqrt{2\pi}}\int_{-\pi/h}^{\pi/h}e^{imh\xi}\hat{u}^{n}(\xi)~d\xi $$

\noindent Combining like terms can simplify the equation:

$$ \frac{1}{\sqrt{2\pi}}\int_{-\pi/h}^{\pi/h}e^{imh\xi}\hat{u}^{n+1}(\xi)\left[1+\frac{a\lambda}{2}e^{1h\xi} - \frac{a\lambda}{2}e^{-ih\xi}\right]~d\xi = 
\frac{1}{\sqrt{2\pi}}\int_{-\pi/h}^{\pi/h}e^{imh\xi}\hat{u}^{n}(\xi)~d\xi$$

\noindent Now we can suggest that since the integrals are equal, their contents are equal as well. Simplifying further, we get:

$$ \hat{u}^{n+1}(\xi) = \frac{1}{1+\frac{a\lambda}{2}e^{ih\xi} - \frac{a\lambda}{2}e^{-ih\xi}} \hat{u}^{n}(\xi) = g(h\xi) \hat{u}^{n+1}(\xi) $$

In order to determine stability, we must set $|g(h\xi)|\leq 0$ and evaluate under what conditions this is true. It is first helpful to replace $h\xi$ with $\theta$, and expand $g(\theta)$ into its sine and cosine parts. This leaves us with

$$ \left|\frac{1}{1+ia\lambda\sin\theta}\right|\leq1 ~~~\longrightarrow ~~~ 1\leq |1+ia\lambda\sin|^2 ~~~ \longrightarrow ~~~ 0\leq a^2\lambda^2\sin^2 ~~~ \longrightarrow |a\lambda\sin\theta| \geq 0 $$

\noindent Since the inequality holds for all values of $\lambda$, the scheme is unconditionally stable.


\section*{2.2.4}
Show that the box scheme
$$\frac{1}{2 k}\left[\left(u_{m}^{n+1}+u_{m+1}^{n+1}\right)-\left(u_{m}^{n}+u_{m+1}^{n}\right)\right]+\frac{a}{2 h}\left[\left(u_{m+1}^{n+1}-u_{m}^{n+1}\right)+\left(u_{m+1}^{n}-u_{m}^{n}\right)\right]=f_{m}^{n}$$

\noindent is consistent with the one-way wave equation $u_{t}+a u_{x}=f$ and is stable for all values of $\lambda$.

\subsubsection*{Solution}

Like the previous solution, we use the Fourier transform substitutes for values of $u$ and plug them into the give scheme. After combining like terms and separating $u^n$ from $u^{n+1}$ values, we get:

$$ 
u_{m}^{n+1}\left[\frac{1}{2} - \frac{a\lambda}{2}\right] + 
u_{m+1}^{n+1}\left[\frac{1}{2} + \frac{a\lambda}{2}\right] = f_{mn} + 
u_{m}^{n}\left[\frac{1}{2} + \frac{a\lambda}{2}\right] + 
u_{m+1}^{n}\left[\frac{1}{2} - \frac{a\lambda}{2}\right] $$

\noindent Now using Fourier transforms:

$$ \frac{1}{\sqrt{2\pi}}\int_{-\pi/h}^{\pi/h}e^{imh\xi}\hat{u}^{n+1}(\xi)\left[\frac{1}{2} - \frac{a\lambda}{2}+e^{-ih\xi}\left(\frac{1}{2} + \frac{a\lambda}{2}\right)\right]~d\xi = $$
$$ f_{mn} + \frac{1}{\sqrt{2\pi}}\int_{-\pi/h}^{\pi/h}e^{imh\xi}\hat{u}^{n}(\xi)\left[\frac{1}{2} + \frac{a\lambda}{2}+e^{-ih\xi}\left(\frac{1}{2} - \frac{a\lambda}{2}\right)\right]~d\xi $$

\noindent Neglecting the $f_{mn}$ term and assuming the integrands are equal, we get:

$$\hat{u}^{n+1}(\xi) = \frac{1 + {a\lambda} + e^{-ih\xi}\left(1 - a\lambda\right)}{1 - {a\lambda} + e^{ih\xi}\left(1+ a\lambda\right)}\hat{u}^n(\xi) = g(h\xi)\hat{u}^n$$

\noindent For stability, we need $|g(h\xi)|\leq1$. Replacing $h\xi$ with $\theta$, we can form  the inequality:

$$\left|1 + {a\lambda} + e^{-\theta}\left(1 - a\lambda\right)\right|^2 \leq \left|1 - {a\lambda} + e^{i\theta}\left(1+ {a\lambda}\right)\right|^2$$

\noindent Squaring the real and non-real values, we get:

$$ [(1+a\lambda)+(1-a\lambda)\cos\theta]^2 + [-(1-a\lambda)\sin\theta]^2 \leq [(1-a\lambda)+(1+a\lambda)\cos\theta]^2 + [(1+a\lambda)\sin\theta]^2$$

\noindent After expanding, this simplifies quite nicely to

$$ (1-a\lambda)^2 \leq (1+a\lambda)^2 $$

\noindent Which is always true for all $a\lambda \geq 0$. Since $a\lambda$ is assumed to be non-negative, this scheme is stable for all $\lambda$.

\section*{3.2.1}
Show that the (forward-backward) MacCormack scheme
$$
{\tilde{u}_{m}^{n+1}=u_{m}^{n}-a \lambda\left(u_{m+1}^{n}-u_{m}^{n}\right)+k f_{m}^{n}} \\$$

$$
{u_{m}^{n+1}=\frac{1}{2}\left(u_{m}^{n}+\tilde{u}_{m}^{n+1}-a \lambda\left(\tilde{u}_{m}^{n+1}-\tilde{u}_{m-1}^{n+1}\right)+k f_{m}^{n+1}\right)}
$$
is a second-order accurate scheme for the one-way wave cquation $(1.1 .1) .$ Show that for $f=0$ it is identical to the Lax-Wendroff scheme $(3.1 .1)$

\subsubsection*{Solution}

To show that a scheme is accurate, we use definition 3.1.1., which states that \textit{a scheme $P_{k, h} u=R_{k, h} f$ that is consistent with the differential equation $P u=f$ is accurate of order $p$ in time and order $q$ in space if for any smooth function $\phi(t, x)$
$$
P_{k . h} \phi-R_{k, h} P \phi=\mathcal{O}\left(k^{\rho}\right)+\mathcal{O}\left(h^{q}\right)
$$
We say that such a scheme is accurate of order $(p, q)$}. 

For this case, we start with defining $P$ and $R$ by expanding and combining like terms, involving a fair amount of algebra when plugging the first step into the second.

\begin{equation*}
    \begin{aligned}
    P_{kh}\phi & = \phi_m^{n+1} - 
    \phi_{m}^{n} + 
    \phi_{m+1}^{n}\left[\frac{1}{2}(a\lambda - a^2\lambda^2 )\right] + 
    \phi_{m-1}^{n}\left[\frac{1}{2}(a\lambda + a^2\lambda^2)\right] \\
     & = f_{m}^{n+1}\left[\frac{1}{2}k\right] +f_{m}^{n}\left[\frac{1}{2}(k-ah)\right] +f_{m-1}^{n}\left[\frac{1}{2}ah\right] \\
     & = R_{kh}f
    \end{aligned}
\end{equation*}

\noindent Now we take a Taylor expansion for each term. The Taylor expansion for each term is give by:

\begin{equation*}
\begin{aligned}
	\phi_m^{n+1} = \phi_m^n + k\phi_t + \mathcal{O}(k^2)\\
	\phi_{m+1}^n = \phi_m^n + h\phi_x + \mathcal{O}(h^2)\\
	\phi_{m-1}^n = \phi_m^n - h\phi_x + \mathcal{O}(h^2)\\
	f_m^{n+1} = f_m^n + kf_t + \mathcal{O}(k^2)\\
	f_{m-1}^n = f_m^n - hf_x + \ \mathcal{O}(h^2)
\end{aligned}
\end{equation*}

\noindent Now, we make a long substitution:

\begin{equation*}
    \begin{aligned}
    P_{kh}\phi = & (\phi_m^n + k\phi_t) - \phi_{m}^{n} + \left(\phi_m^n + h\phi_x  \right)\left[\frac{1}{2}(a\lambda - a^2\lambda^2)\right] +...\\
    &...+\left(\phi_m^n - h\phi_x \right)\left[\frac{1}{2}(a\lambda + a^2\lambda^2)\right] +\mathcal{O}(k^2) + \mathcal{O}(h^2)\\
      = &\left(f_m^n + kf_t \right)\left[\frac{1}{2}k\right] + f_{m}^{n}\left[\frac{1}{2}(k-ah)\right] +\left(f_m^n - hf_x \right)\left[\frac{1}{2}ah\right] + \mathcal{O}(k^2) + \mathcal{O}(h^2)\\
      =& R_{kh}f
    \end{aligned}
\end{equation*}

\noindent Now we have
$$ P_{kh}\phi = k\phi_t + ak\phi_{x} + \mathcal{O}(k^2) + \mathcal{O}(h^2)  ~~~ \longrightarrow ~~~ P_{kh}\phi = k\phi_t + ak\phi_{x} + \mathcal{O}(k^2) + \mathcal{O}(h^2)$$

\noindent and
$$ R_{kh}f = kf + \frac{1}{2}k^2f_t  - \frac{1}{2}a\lambda khf_x  + \mathcal{O}(k^2) + \mathcal{O}(h^2) $$

\noindent Now we can make the substitutions for the one-way wave equation
$$ f = \phi_t + a\phi_x ~~~~~~ f_t = \phi_{tt} + a\phi_{xt} ~~~~~~  f_x = \phi_{tx} + a\phi_{xx} $$

\noindent This leads to
$$ R_{kh}P\phi = k(\phi_t + a\phi_x) + \frac{1}{2}k^2(\phi_{tt} + a\phi_{xt})  - \frac{1}{2}ah^2(\phi_{tx} + a\phi_{xx})  + \mathcal{O}(k^2) + \mathcal{O}(h^2)$$

$$ \longrightarrow  R_{kh}P\phi = k\phi_t + ka\phi_x + \mathcal{O}(k^2) + \mathcal{O}(h^2)$$

\noindent This matches our definition. Since $P_{kh}\phi = R_{kh}O\phi$, the scheme is second order accurate.

If $f=0$, then this scheme would look like:

$$
u_{m}^{n+1}=\frac{1}{2}\left\{u_{m}^{n}+\left[u_{m}^{n}-a \lambda(u_{m+1}^{n}-u_{m}^{n})\right]   (1 - a\lambda)  -a \lambda\left[-(u_{m-1}^{n}-a \lambda(u_{m}^{n}-u_{m-1}^{n}))\right]\right\} 
$$
$$\longrightarrow u_m^{n+1} = 
    u_{m}^{n}\left[1-a^2\lambda^2\right] + 
    u_{m+1}^{n}\left[\frac{1}{2}(a^2\lambda^2 -a\lambda)\right] + 
    u_{m-1}^{n}\left[\frac{1}{2}(a\lambda + a^2\lambda^2)\right]  $$


\noindent And compared with the Lax-Wendroff scheme, which is
$$
\begin{aligned}
u_{m}^{n+1}=& u_{m}^{n}-\frac{a \lambda}{2}\left(u_{m+1}^{n}-u_{m-1}^{n}\right)+\frac{a^{2} \lambda^{2}}{2}\left(u_{m+1}^{n}-2 u_{m}^{n}+u_{m-1}^{n}\right)
\end{aligned}
$$
$$\longrightarrow u_m^{n+1} = u_m^n\left[1-a^2\lambda^2\right] + u_{m+1}^n\left[\frac{a^2\lambda^2}{2} - \frac{a\lambda}{2}\right] + u_{m-1}^n\left[\frac{a\lambda}{2} + \frac{a^2\lambda^2}{2}\right] $$

\noindent which are equal in this instance.

\section*{3.2.3}
Show that the box scheme
$$
\frac{1}{2 k}\left[\left(v_{m}^{n+1}+v_{m+1}^{n+1}\right)-\left(v_{m}^{n}+v_{m+1}^{n}\right)\right]
\quad+\frac{a}{2 h}\left[\left(v_{m+1}^{n+1}-v_{m}^{n+1}\right)+\left(v_{m+1}^{n}-v_{m}^{n}\right)\right]
=$$
$$\frac{1}{4}\left(f_{m+1}^{n+1}+f_{m}^{n+1}+f_{m+1}^{n}+f_{m}^{n}\right)$$
is an approximation to the one-way wave equation $u_{t}+a u_{x}=f$ that is accurate of order $(2,2)$ and is stable for all values of $\lambda$

\subsubsection*{Solution}

This is going to be a long one. First, I find it easiest to define $P_{kh}\phi$ and $R_{kh}f$ first:

\begin{equation*}
    \begin{aligned}
    P_{kh}\phi & = \phi_m^{n+1}[1-a\lambda] + 
    \phi_{m+1}^{n+1}[1+a\lambda] + 
    \phi_{m}^{n}[-1-a\lambda] + 
    \phi_{m+1}^{n}[-1+a\lambda] \\
     & = f_{m}^{n+1}\left[\frac{k}{2}\right] +f_{m+1}^{n+1}\left[\frac{k}{2}\right] +f_{m}^{n}\left[\frac{k}{2}\right] f_{m+1}^{n}\left[\frac{k}{2}\right] = R_{kh}f
    \end{aligned}
\end{equation*}

\noindent Now, for the Taylor expansion:

\begin{equation*}
\begin{aligned}
	&\phi_m^{n+1} = \phi_m^n + k\phi_t + \mathcal{O}(k^2)\\
	&\phi_{m+1}^{n+1} = \phi_m^n + k\phi_t + h\phi_x+ \mathcal{O}(k^2)+ \mathcal{O}(h^2)\\
	&\phi_{m+1}^n = \phi_m^n + h\phi_x + \mathcal{O}(h^2)\\
	&f_m^{n+1} = f_m^n + kf_t + \mathcal{O}(k^2)\\
	&f_{m+1}^{n+1} = f_m^n + kf_t + +hf_x +\mathcal{O}(k^2)+ \mathcal{O}(h^2)\\
	&f_{m+1}^{n} = f_m^n + hf_x + \mathcal{O}(h^2)\\
\end{aligned}
\end{equation*}

\begin{equation*}
    \begin{aligned}
    &(\phi_m^n + k\phi_t)[1-a\lambda] + 
    (\phi_m^n + k\phi_t + h\phi_x)[1+a\lambda] + 
    \phi_{m}^{n}[-1-a\lambda] + 
    (\phi_m^n + h\phi_x)[-1+a\lambda] \\
     &= (f_m^n + kf_t)\left[\frac{k}{2}\right] +
     (f_m^n + kf_t + hf_x)\left[\frac{k}{2}\right] +
     f_{m}^{n}\left[\frac{k}{2}\right] 
     (f_m^n + hf_x)\left[\frac{k}{2}\right]+\mathcal{O}(k^2)+ \mathcal{O}(h^2)
    \end{aligned}
\end{equation*}

\noindent which simplifies to
$$ (2k)\phi_t + (2ka)\phi_x = (2k)f + (k^2)f_t + (hk)f_x +\mathcal{O}(k^2)+ \mathcal{O}(h^2)$$

\noindent Plugging this into $P_{kh}\phi - R_{kh}P\phi$ for the one-way wave equation yields:
$$ (k^2)\phi_{tt} + (hka)\phi_{xx} + (hk)\phi_{tx}+ (k^2a)\phi_{xt}= \mathcal{O}(k^2)+ \mathcal{O}(h^2)$$

\noindent which we can see is solely comprised of higher order terms, this showing that this scheme has second order accuracy.


\pagebreak
\section*{3.4.1 (Numerical)}
Solve the initial-boundary value problem $(1.2 .1)$ with the leapfrog scheme and the following boundary conditions. Use $a=1 .$ Only (d) should give good results. Why?

(a) At $x=0,$ specify $u(t, 0) ;$ at $x=1,$ use boundary condition $(3.4 .1 \mathrm{b})$

(b) At $x=0,$ specify $u(t, 0) ;$ at $x=1,$ specify $u(t, 1)=0$

(c) At $x=0,$ use boundary condition $(3.4 .1 \mathrm{b}) ;$ at $x=1,$ use $(3.4 .1 \mathrm{c})$

(d) At $x=0,$ specify $u(t, 0) ;$ at $x=1,$ use boundary condition $(3.4 .1 \mathrm{c})$

\subsubsection*{Solution}

Problem 1.2.1 states:

Consider system (1.2.2) on the interval [0, 1], with $a$ equal to 0 and $b$ equal to 1 and with the boundary conditions $u^{1}$ equal to 0 at the left and $u^{1}$ equal to 1 at the right boundary. Show that if the initial data are given by $u^{1}(0, x)=x$ and $u^{2}(0, x)=1,$ then the solution is $u^{1}(t, x)=x$ and $u^{2}(t, x)=1-t$ for all $(t, x)$ with $0 \leq x \leq 1$ and $0 \leq t$

System (1.2.2) is given by:

$$
\left(\begin{array}{l}
u^{1} \\
u^{2}
\end{array}\right)_{t}+\left(\begin{array}{ll}
1 & 1 \\
1 & 1
\end{array}\right)\left(\begin{array}{l}
u^{1} \\
u^{2}
\end{array}\right)_{x}=0
$$

And the leapfrog scheme is given by:

$$
\frac{u_{m}^{n+1}-u_{m}^{n-1}}{2 k}+a \frac{u_{m+1}^{n}-u_{m-1}^{n}}{2 h}=0
$$


\end{document}

\subsubsection*{Solution}